\documentclass[aspectratio=169]{beamer}
\usetheme{AnnArbor}

\usepackage[utf8]{inputenc}
\usepackage[T1]{fontenc}
\usepackage[french]{babel}
\usepackage{graphicx}
\usepackage{lmodern}

\title{L'Intelligence Artificielle et les Métiers du Futur}
\author{Imrann ASEERVATHAM}
\institute{3B, Collège des Touleuses}
\date{02/06/2025}

\begin{document}

\maketitle

\begin{frame}{Sommaire}
\tableofcontents
\end{frame}

\section{L’Intelligence Artificielle}
\begin{frame}
\begin{block}{Définition}
L'intelligence artificielle désigne des \textbf{technologies capables} d'imiter des \textbf{capacités humaines}, comme par exemple apprendre ou prendre une décision \textbf{soi-même}.
\end{block}
\vspace{0.5cm}
\begin{block}{Rôle de l'IA}
L'intelligence artificielle permet à un système d'apprentissage d'exécuter des tâches humaines associées à l'intelligence.
\end{block}
\end{frame}

\begin{frame}
\centering
\includegraphics[height=0.8\textheight]{azert.png}
\end{frame}

\section{Les secteurs impactés par l'IA}
\begin{frame}
\begin{itemize}
\item \textbf{Médecine}: détection de maladies, aide au diagnostic, chirurgie assistée par robot, analyse d’imagerie médicale.
\item \textbf{Transports}: voitures autonomes, gestion du trafic, optimisation des livraisons (Uber Eats, Amazon…).
\item \textbf{Éducation}: cours personnalisés, soutien scolaire avec des IA (comme ChatGPT, Khan Academy).
\item \textbf{Industrie}: contrôle de qualité, gestion des machines, optimisation des chaînes de production.
\item \textbf{Finance}: détection de fraude, analyse de données boursières, assistance à l’investissement.
\item \textbf{Sécurité et défense}: surveillance intelligente, cybersécurité, prévention des menaces.
\end{itemize}
\end{frame}

\section{The Impact of AI on Future Jobs}
\begin{frame}{Jobs of the Future Thanks to AI}
\begin{itemize}
\item \textbf{AI Engineer} – Designs and improves artificial intelligence systems.
\item \textbf{Data Scientist} – Analyzes large data sets to help companies make smart decisions.
\item \textbf{Machine Learning Specialist} – Creates algorithms that help machines learn from experience.
\item \textbf{Robot Technician} – Builds and maintains intelligent robots for factories, hospitals, etc.
\item \textbf{Cybersecurity Analyst} – Uses AI to detect and prevent cyberattacks.
\item \textbf{Virtual Assistant Developer} – Creates smart assistants like Siri, Alexa, or ChatGPT.
\end{itemize}
\end{frame}

\begin{frame}{Jobs Disappearing Due to AI}
\begin{itemize}
\item \textbf{Cashiers} – AI takes payments. No need for cashiers.
\item \textbf{Truck Drivers} – AI drives trucks. No human needed.
\item \textbf{Journalists} – AI writes news. Fewer jobs for writers.
\item \textbf{Factory Workers} – Robots build things. Fewer workers needed.
\item \textbf{Software Engineers} – AI writes simple code. Humans do the complex part.
\end{itemize}
\end{frame}

\section{Conclusion}
\begin{frame}{Conclusion}
\begin{block}{Why this topic?}
\begin{itemize}
\item A passion for artificial intelligence
\item A major impact on the future
\item A desire to understand future job transformations
\end{itemize}
\end{block}

\vspace{0.5cm}

\begin{block}{My Career Goals}
\begin{itemize}
\item An interest in working in technology and AI
\item Becoming an AI engineer or researcher
\item A topic that prepares for tomorrow’s careers
\end{itemize}
\end{block}
\end{frame}

\begin{frame}
\centering
\vfill
{\Huge Merci de m’avoir écouté}
\\
\vspace{1cm}
{\Huge Thank you for listening}
\vfill
\end{frame}

\end{document}
